\documentclass[a4paper, 10pt, american]{article}

% useful packages
\usepackage{lipsum}
\usepackage{cite}
\usepackage{url}
\usepackage[title]{appendix}
\usepackage[margin=1in]{geometry} % set 1in margins

% correct bad hyphenation here
\hyphenation{op-tical net-works semi-conduc-tor}


\begin{document}

\title{Kyoto VR MQP Paper First Draft}

\author{William~Campbell, Cole~Granof, and Joseph~Petitti}

\maketitle

\begin{abstract}
Abstract goes here. Their early work was a little too new wave for my taste. But
when \textit{Sports} came out in '83, I think they really came into their own,
commercially and artistically. The whole album has a clear, crisp sound, and a
new sheen of consummate professionalism that really gives the songs a big boost.
He's been compared to Elvis Costello, but I think Huey has a far more bitter,
cynical sense of humor. In '87, Huey released this; \textit{Fore!}, their most
accomplished album. I think their undisputed masterpiece is ``Hip To Be
Square.'' A song so catchy, most people probably don't listen to the lyrics. But
they should, because it's not just about the pleasures of conformity and the
importance of trends. It's also a personal statement about the band itself. Hey,
Paul!
\end{abstract}

\section{Introduction}
\label{sec:introduction}

This is a sample paragraph that exists as a placeholder because
we haven't written the introduction yet. When we do, this paragraph should be
removed.

\lipsum[1]

\section{Background}
\label{sec:background}

As smartphones and mobile technology becomes more prevalent,
new forms of human-computer interaction are becoming mainstream. Smartphones
allow for an unprecedented degree of connectivity with the digital world, but
can also serve as a tool for enhancing the physical world. In this section we
explain the origins and uses of some of this technology.

\subsection{What is Augmented Reality?}
\label{sec:whatIsAugmentedReality}

Augmented Reality, or AR, is a type of human-computer interface where
perceptions of the real world are enhanced by computer-generated information.
This differs from Virtual Reality (VR), in that a VR experience consists
exclusively of virtual information. In AR, virtual information is mixed with
sensory input from the real world \cite{carmigniani2011}. This can enhance the
user's perception of reality by providing information that would be difficult or
impossible to display through traditional means.

For example, AR can be used to display information about historical events,
places, and objects overlaid onto images of the real world \cite{saenz2009}.
This provides the user with useful information without needing to alter the real
historic site.

\subsubsection{Current Augmented Reality Technology}
\label{sec:currentAugmentedRealityTechnology}

While preparing for this project our team researched the current state of
augmented reality technology. Smartphones are the most commonly used AR hardware
by far. Typically smartphone AR applications make use of the phone's camera,
accelerometer, gyroscope, and GPS sensors to reproduce a view of the real world
with virtual information layered on top of it \cite{bonsor2018}.

\subsection{Augmented Reality Use Cases}
\label{sec:augmentedRealityUseCases}

\lipsum[1]

\subsection{Challenges of Augmented Reality}
\label{sec:challengesOfAugmentedReality}

\lipsum[1]

\subsection{Apps for Art and Culture}
\label{sec:appsForArtAndCulture}

\lipsum[2-3]

\subsubsection{izi.TRAVEL}
\label{sec:iziTravel}

\lipsum[4-5]

\subsection{Platforms}
\label{sec:platforms}

\subsubsection{ARCore and ARKit}
\label{sec:ARCoreAndARKit}

\subsubsection{Wikitude}
\label{sec:wikitude}

\subsubsection{KudanSlam}
\label{sec:kudanSlam}

\subsubsection{Unity}
\label{sec:unity}

\subsubsection{ViroReact}
\label{sec:viroReact}

\subsubsection{motive.io}
\label{sec:motive.io}

\section{Kyoto VR Ecosystem}
\label{sec:KyotoVREcosystem}

\lipsum

\section{Implementation and Technology}
\label{sec:implementationAndTechnology}

\section{Testing}
\label{sec:testing}

\section{Conclusion}
\label{sec:conclusion}

Conclusions are typically hard to write, so we'll save it for last. This is
once again just a placeholder because we haven't written this section yet.

\lipsum[1]

\clearpage % references should be on their own page

\bibliographystyle{IEEEtran}
\bibliography{references}

\appendices
\section{Proof of the First Zonklar Equation}
Appendix one text goes here.

\section{}
Appendix two text goes here.


\end{document}


